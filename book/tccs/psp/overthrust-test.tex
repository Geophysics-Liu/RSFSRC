\begin{table}
\footnotesize
\centering
\begin{tabular}{r|c|c|c|c|c|}
\cline{2-6}
 & \multicolumn{5}{|c|}{number of processes} \\
\cline{2-6}
 & 128 & 256 & 512 & 1024 & 2048  \\
\hline
\multicolumn{1}{|r|}{Hz}
 & 3.175
 & 4
 & 5.04
 & 6.35
 & 8 \\
\hline
\multicolumn{1}{|r|}{grid}
 & $319^2 \!\times\! 75$
 & $401^2 \!\times\! 94$
 & $505^2 \!\times\! 118$
 & $635^2 \!\times\! 145$
 & $801^2 \!\times\! 187$ \\
\hline
\multicolumn{1}{|r|}{setup (sec)}
 & 48.40 (46.23)
 & 66.33 (63.41) 
 & 92.95 (86.90)
 & 130.4 (120.6)
 & 193.0 (175.2) \\
\hline
\multicolumn{1}{|r|}{apply (sec/rhs)}
 & 0.468 (1.07)
 & 0.550 (1.28) 
 & 0.645 (2.40)
 & 0.700 (3.33)
 & 0.880 (6.13) \\
\hline
\multicolumn{1}{|r|}{3 digits (iter's)}
 & 42
 & 44
 & 42
 & 39
 & 40 \\
\hline
\multicolumn{1}{|r|}{4 digits (iter's)}
 & 54
 & 57
 & 57
 & 58
 & 58 \\
\hline
\multicolumn{1}{|r|}{5 digits (iter's)}
 & 63
 & 69
 & 70
 & 68
 & 72 \\
\hline
\end{tabular}
\caption{Convergence rates and timings on TACC's Lonestar for the 
SEG/EAGE Overthrust model, where timings in parentheses do not make use of 
selective inversion. All cases used a double-precision second-order 
stencil with five grid spacings for all PML (with an amplitude
of 7.5), and a damping parameter of $2.25 \pi$.
The preconditioner was configured with four planes per panel and eight 
processes per node, and the `apply' timings are with respect to a single 
application of the preconditioner to four right-hand sides.}
\label{tbl:overthrust-test}
\end{table}
