\append{Second-order reflection traveltime derivatives}
%%%%%%%%%%%%%%%%%%%%%%%%%%%%%%%%%%%%%%%%%%%%%%%%%%%%%%%
\label{chapter:deriv}
This appendix contains a derivation of equations connecting
second-order partial derivatives of the reflection traveltime with the
geometric properties of the reflector in a constant velocity medium.
These equations are used in the main text of this paper to describe
the amplitude behavior of offset continuation.  Let $\tau(s,r)$ be the
reflection traveltime from the source $s$ to the receiver $r$.
Consider a formal equality
\begin{equation}
\tau(s,r)=\tau_1\left(s,x(s,r)\right)+\tau_2\left(x(s,r),r\right)\;,   
\label{eqn:t1pt2} 
\end{equation}
where $x$ is the reflection point parameter, $\tau_1$ corresponds to the
incident ray, and $\tau_2$ corresponds to the reflected ray.
Differentiating (\ref{eqn:t1pt2}) with respect to $s$ and $r$ yields
\begin{eqnarray}
{\partial \tau \over \partial s}  & = & 
{\partial \tau_1 \over \partial s} + {\partial \tau \over \partial x}\,
{\partial x \over \partial s}\;,
\label{eqn:dt1ds} \\
{\partial \tau \over \partial r}  & = & 
{\partial \tau_2 \over \partial r} + {\partial \tau \over \partial x}\,
{\partial x \over \partial r}\;.
\label{eqn:dt2dr} 
\end{eqnarray}
According to Fermat's principle, the two-point reflection ray path must
correspond to  the traveltime stationary point. Therefore 
\begin{equation}
{\partial \tau \over \partial x} \equiv 0  
\label{eqn:fermat} 
\end{equation}
for any $s$ and $r$. Taking into account (\ref{eqn:fermat}) while
differentiating (\ref{eqn:dt1ds}) and (\ref{eqn:dt2dr}), we get
\begin{eqnarray}
{\partial^2 \tau \over \partial s^2}  & = & 
{\partial^2 \tau_1 \over \partial s^2} + 
B_1\,
{\partial x \over \partial s}\;,
\label{eqn:d2tds2} \\
{\partial^2 \tau \over \partial r^2}  & = & 
{\partial^2 \tau_2 \over \partial r^2} + 
B_2\,
{\partial x \over \partial r}\;,
\label{eqn:d2tdr2} \\
{\partial^2 \tau \over \partial s \partial r}  & = &  
B_1\,
{\partial x \over \partial r}\;=
B_2\,
{\partial x \over \partial s}\;,
\label{eqn:d2tdsdr}
\end{eqnarray}
where
\[
B_1={\partial^2 \tau_1 \over \partial s \partial x}\;;\;
B_2={\partial^2 \tau_2 \over \partial r \partial x}\;.
\]
Differentiating equation~(\ref{eqn:fermat}) gives us the additional
pair of equations
\begin{eqnarray}
C\,{\partial x \over \partial s}+B_1  & = & 0\;,
\label{eqn:b12c} \\
C\,{\partial x \over \partial r}+B_2  & = & 0\;,
\label{eqn:b22c}
\end{eqnarray}
where
\[
C={\partial^2 \tau \over \partial x^2}=
{\partial^2 \tau_1 \over \partial x^2}+
{\partial^2 \tau_2 \over \partial x^2}\;.
\]
Solving the system (\ref{eqn:b12c}) - (\ref{eqn:b22c}) for $\partial x
\over \partial s$ and $\partial x \over \partial r$ and substituting
the result into (\ref{eqn:d2tds2}) - (\ref{eqn:d2tdsdr}) produces the
following set of expressions:
\begin{eqnarray}
{\partial^2 \tau \over \partial s^2}  & = & 
{\partial^2 \tau_1 \over \partial s^2} -
C^{-1}\,B_1^2\;;
\label{eqn:cs2} \\
{\partial^2 \tau \over \partial r^2}  & = & 
{\partial^2 \tau_2 \over \partial r^2} -
C^{-1}\,B_2^2\;;
\label{eqn:cr2} \\
{\partial^2 \tau \over \partial s \partial r}  & = &  
- C^{-1}\,B_1\,B_2\;.
\label{eqn:csr}
\end{eqnarray}
In the case of a constant velocity medium, expressions (\ref{eqn:cs2}) to
(\ref{eqn:csr}) can be applied directly to the explicit
equation for the two-point eikonal 
\begin{equation}
\tau_1(y,x)=\tau_2(x,y)={\sqrt{(x-y)^2+z^2(x)}\over v}\;.
\label{eqn:twopoint}
\end{equation}
Differentiating (\ref{eqn:twopoint}) and taking into account the trigonometric
relationships for the incident and reflected rays (Figure
\ref{fig:ocoray}), one can 
evaluate all the quantities in (\ref{eqn:cs2}) to (\ref{eqn:csr}) explicitly.
After some heavy algebra, the resultant expressions for the traveltime 
derivatives take the form 
\begin{eqnarray}
{\partial \tau \over \partial s}  = 
{\partial \tau_1 \over \partial s} =
{\sin{\alpha_1}\over v}& \;;\; &
{\partial \tau \over \partial r} =
{\partial \tau_2 \over \partial r} =
{\sin{\alpha_2}\over v}\;;
\label{eqn:tstr} \\
{\partial \tau_1 \over \partial x}  = 
{\sin{\gamma}\over v \cos{\alpha}} & \;;\; &
{\partial \tau_2 \over \partial x} =
- {\sin{\gamma}\over v \cos{\alpha}}\;;
\label{eqn:txx}
\end{eqnarray}
\begin{eqnarray}
B_1  & = & 
{\partial^2 \tau_1 \over \partial s\,\partial x} =
{\cos{\alpha_1}\over{v\,D\,\cos{\alpha}}}\,
\left(-1-{\sin{\gamma}\over\cos{\alpha}}\,\sin{\alpha_1}\right)\;;
\label{eqn:B1} \\ 
B_2  & = &
{\partial^2 \tau_2 \over \partial r\,\partial x} =
{\cos{\alpha_2}\over{v\,D\,\cos{\alpha}}}\,
\left(-1+{\sin{\gamma}\over\cos{\alpha}}\,\sin{\alpha_2}\right)\;;
\label{eqn:B2} 
\end{eqnarray}
\begin{equation}
B_1\,B_2  =  {\cos^6{\gamma}\over v^2\,D^2\,a^4}\;;\;
B_1+B_2 = -2\,{\cos^3{\gamma}\over v\,D\,a^2}\,\left(2\,a^2-1\right)\;;
\label{eqn:B1B2}
\end{equation}
\begin{equation}
{\partial^2 \tau_1 \over \partial x^2} =
{{\cos^2{\gamma}+D\,K}\over{v\,D\,\cos^3{\alpha}}}\,\cos{\alpha_1}\;;\;
{\partial^2 \tau_2 \over \partial x^2} =
{{\cos^2{\gamma}+D\,K}\over{v\,D\,\cos^3{\alpha}}}\,\cos{\alpha_2}\;;
\label{eqn:C1C2} 
\end{equation}
\begin{equation}
C={\partial^2 \tau_1 \over \partial x^2}+{\partial^2 \tau_2 \over \partial x^2}=
2\,\cos{\gamma}\,{{\cos^2{\gamma}+D\,K}\over{v\,D\,\cos^3{\alpha}}}\;.
\label{eqn:C}
\end{equation}
Here $D$ is the length of the normal (central) ray, $\alpha$ is its dip angle
($\alpha={{\alpha_1+\alpha_2}\over 2}$, $\tan{\alpha}=z'(x)$),
$\gamma$ is the reflection angle 
$\left(\gamma={{\alpha_2-\alpha_1}\over 2}\right)$, $K$ is the reflector 
curvature at the reflection point $\left(K=z''(x)\,\cos^3{\alpha}\right)$, and 
$a$ is the dimensionless function of $\alpha$ and $\gamma$ defined in (\ref{eqn:A}).

The equations derived in this appendix were used to obtain the equation
\begin{equation}
\tau_n\,\left({\partial^2 \tau_n \over \partial y^2}-
{\partial^2 \tau_n \over \partial h^2}\right)=
4\,\left(\tau\,{\partial^2 \tau \over \partial s\,\partial r}+
{\cos^2{\gamma}\over v^2}\right)=
4\,
{\cos^2{\gamma}\over v^2}\,\left({\sin^2{\alpha}+DK}\over
{\cos^2{\gamma}+DK}\right)\;,
\label{eqn:curved}
\end{equation}
which coincides with (\ref{eqn:curve}) in the main text.

\append{The kinematics of offset continuation}
%%%%%%%%%%%%%%%%%%%%%%%%%%%%%%%%%%%%%%%%%
\label{chapter:kinem}
This Appendix presents an alternative method to derive equation
(\ref{eqn:summation}), which describes the summation path of the
integral offset continuation operator. The method is based on the
following considerations.

The summation path of an integral (stacking) operator coincides with
the phase function of the impulse response of the inverse operator.
Impulse response is by definition the operator reaction to an impulse
in the input data. For the case of offset continuation, the input is a
reflection common-offset gather. From the physical point of view, an
impulse in this type of data corresponds to the special focusing
reflector (elliptical isochrone) at the depth. Therefore, reflection
from this reflector at a different constant offset corresponds to the
impulse response of the OC operator.  In other words, we can view
offset continuation as the result of cascading prestack common-offset
migration, which produces the elliptic surface, and common-offset
modeling (inverse migration) for different offsets.  This approach
resemble that of \cite{GPR29-03-03740406}.  It was also applied to a
more general case of azimuth moveout (AMO) by
\cite{SEG-1995-1449} and fully generalized by 
\cite{GPR48-01-01350162}.  The geometric approach implies that
in order to find the summation pass of the OC operator, one should
solve the kinematic problem of reflection from an elliptic reflector
whose focuses are in the shot and receiver locations of the output
seismic gather.

In order to solve this problem , let us consider an elliptic surface of
the general form
\begin{equation}
h(x)=\sqrt{d^2-\beta\,(x-x')^2}\;,
\label{eqn:ellips}
\end{equation}
where $0 < \beta < 1$. In a constant velocity medium, the reflection
ray path for a given source-receiver pair on the surface is controlled
by the position of the reflection point $x$.  Fermat's principle
provides a required constraint for finding this position. According to
Fermat's principle, the reflection ray path corresponds to a
stationary value of the travel-time. Therefore, in the neighborhood of
this path,
\begin{equation}
{\partial \tau(s,r,x) \over \partial x} = 0\;,
\label{eqn:fermat1}
\end{equation}
where $s$ and $r$ stand for the source and receiver locations on the
surface, and $\tau$ is the reflection traveltime
\begin{equation}
\tau(s,r,x) =   { \sqrt{h^2(x)+(s-x)^2} \over v} + 
                { \sqrt{h^2(x)+(r-x)^2} \over v}\;.
\label{eqn:length1}
\end{equation}

Substituting equations~(\ref{eqn:length1}) and (\ref{eqn:ellips}) into
(\ref{eqn:fermat1}) leads to a quadratic algebraic equation on the
reflection point parameter $x$.  This equation has the explicit
solution
\begin{equation}
x(s,r)= x' + {{\xi^2+H^2-h^2+\mbox{sign}(h^2-H^2)\,
\sqrt{\left(\xi^2-H^2-h^2\right)^2-4\,H^2\,h^2}\over
{2\,\xi\,(1-\beta)}}}\;,
\label{eqn:reflection}
\end{equation}
where $h=(r-s)/2$, $\xi = y-x'$, $y=(s+r)/2$, and $H^2=d^2\,\left({1
    \over \beta} - 1\right)$. Replacing $x$ in equation
(\ref{eqn:length1}) with its expression (\ref{eqn:reflection}) solves
the kinematic part of the problem, producing the explicit traveltime
expression
\begin{equation}
\tau(s,r)=\left\{
        \begin{array}{lcr}\displaystyle{
{1 \over v} \sqrt{{4\,h^2-\beta\,(f+g)^2} \over {1-\beta}}}
& \mbox{for} & h^2 > H^2 \\ & & \\ \displaystyle{
{1 \over v} \sqrt{{4\,h^2+\beta\,(F+G)^2} \over {1-\beta}}}
& \mbox{for} & h^2 < H^2
        \end{array}
        \right.\;, 
\label{eqn:tau}
\end{equation} 
where 
\begin{eqnarray}
f=\sqrt{(r-x')^2-H^2}\; & , & \;g=\sqrt{(s-x')^2-H^2}\;,
\nonumber \\
F=\sqrt{H^2-(r-x')^2}\; & , & \;G=\sqrt{H^2-(s-x')^2}\;.
\nonumber
\end{eqnarray}

The two branches of equation~(\ref{eqn:tau}) correspond to the
difference in the geometry of the reflected rays in two different
situations. When a source-and-receiver pair is inside the focuses of
the elliptic reflector, the midpoint $y$ and the reflection point $x$
are on the same side of the ellipse with respect to its small
semi-axis. They are on different sides in the opposite case (Figure
\ref{fig:ell}).

\inputdir{Sage}

\sideplot{ell}{height=1.in}{.}{
Reflections from an ellipse. The three pairs of reflected rays
correspond to a common midpoint (at 0.1) and different offsets. The
focuses of the ellipse are at 1 and -1.
} 

If we apply the NMO correction, equation (\ref{eqn:tau}) is transformed to
\begin{equation}
\tau_n(s,r)=\left\{
        \begin{array}{lcr}\displaystyle{
{1 \over v} \sqrt{\beta \over {1-\beta}}\,
\sqrt{4\,h^2-(f+g)^2}}
& \mbox{for} & h^2 > H^2 \\ & & \\ \displaystyle{
{1 \over v} \sqrt{\beta \over {1-\beta}}\,
\sqrt{4\,h^2+(F+G)^2}}
& \mbox{for} & h^2 < H^2
        \end{array}
        \right.\;. 
\label{eqn:taun}
\end{equation}
Then, recalling the relationships between the parameters of the
focusing ellipse $r$, $x'$ and $\beta$ and the parameters of the
output seismic gather \cite[]{GPR29-03-03740406}

\begin{equation}
r={ {v\,t_n} \over 2}\;,\;x'=y\;,\;
\beta={t_n^2 \over {t_n^2+{{4\,h^2} \over v^2}}}\;,\;
H=h\;,
\label{eqn:ell2}
\end{equation}
and substituting expressions (\ref{eqn:ell2}) into equation (\ref{eqn:taun}) yields the
expression
\begin{equation}
t_1(s_1,r_1;s,r,t_n)=\left\{
        \begin{array}{lcr}\displaystyle{
{t_n \over {2\,h}}\,
\sqrt{4\,h_1^2-(f+g)^2}}
& \mbox{for} & h_1^2 > h^2 \\ & & \\ \displaystyle{
{t_2 \over {2\,h}}\,
\sqrt{4\,h_1^2+(F+G)^2}}
& \mbox{for} & h_1^2 < h^2
        \end{array}
        \right.\;, 
\label{eqn:final}
\end{equation}
where 
\begin{eqnarray}
f=\sqrt{(r_1-r)\,(r_1-s)}\;,\;g=\sqrt{(s_1-r)\,(s_1-s)}\;,
\nonumber \\
F=\sqrt{(r-r_1)\,(r_1-s)}\;,\;G=\sqrt{(s_1-r)\,(s-s_1)}\;.
\nonumber
\end{eqnarray}

It is easy to verify algebraically the mathematical equivalence of
equation (\ref{eqn:final}) and equation (\ref{eqn:summation}) in the
main text. The kinematic approach described in this appendix applies
equally well to different acquisition configurations of the input and
output data. The source-receiver parameterization used in
(\ref{eqn:final}) is the actual definition for the summation path of
the integral shot continuation operator
\cite[]{SEG-1993-0673,GEO61-06-18461858}. A family of these summation
curves is shown in Figure \ref{fig:shc}.

\plot{shc}{width=6in,height=3in}{.}{
Summation paths of the integral shot continuation. The output source
is at -0.5 km. The output receiver is at 0.5 km. The indexes of the
curves correspond to the input source location.}


