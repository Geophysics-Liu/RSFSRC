\begin{abstract}

Stacking operators are widely used in seismic imaging and seismic data
processing. Examples include Kirchhoff datuming, migration, offset
continuation, DMO, and velocity transform. Two primary approaches
exist for inverting such operators. The first approach is iterative
least-squares optimization, which involves the construction of the adjoint
operator. The second approach is asymptotic inversion, where an approximate
inverse operator is constructed in the high-frequency asymptotics.  Adjoint
and asymptotic inverse operators share the same kinematic properties, but
their amplitudes (weighting functions) are defined differently. This paper
describes a theory for reconciling the two approaches. I introduce a pair of
the {\em asymptotic pseudo-unitary} operators, which possess both the property
of being adjoint and the property of being asymptotically inverse. The
weighting function of the asymptotic pseudo-unitary stacking operators is
shown to be completely defined by the derivatives of the operator
kinematics. I exemplify the general theory by considering several particular
examples of stacking operators. Simple numerical experiments demonstrate a
noticeable gain in efficiency when the asymptotic pseudo-unitary operators are
applied for preconditioning iterative least-squares optimization.

\end{abstract}
