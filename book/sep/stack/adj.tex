
\section{LEAST-SQUARES INVERSION AND ADJOINT OPERATORS}
%%%%%%%%%%%%%%%%%%%%%
Least-squares inversion is widely used in practice not only because it
  is applicable even when the asymptotic results are unavailable but also
  because of its ability to handle finite sampling effects that are 
  difficult to handle in asymptotic theory \cite[]{GEO65-05-13641371}.

The theoretical least-squares inverse of operator (\ref{eqn:operator}) has the
well-known form \cite[]{tarantola}
\begin{equation}
\widetilde{M}(z,x)={\bf \widetilde{A}}[S(t,y)]=
{\bf \left(A^{T}\,A\right)^{\dagger}\,A^{T}}[S(t,y)]\;,
\label{eqn:LS}
\end{equation} 
where $\dagger$ denotes pseudo-inverse, and 
the adjoint operator ${\bf A^{T}}$ is defined by the dot-product
test:
\begin{equation}
\left(S(t,y),{\bf A}[M(z,x)]\right) \equiv 
\left({\bf A^{T}}[S(t,y)],M(z,x)\right)\;.
\label{eqn:dottest}
\end{equation}
With a specified definition of the dot-product, the generalized
inverse minimizes the following quantity, which is the squared $L_2$
norm of the residual:
\begin{equation}
\left(S(t,y)-{\bf A}[M(z,x)],
S(t,y)-{\bf A}[M(z,x)]\right)\;.
\end{equation}
In the case of integral operators, a natural definition of the dot-product
is the double integral
\begin{equation}
\left(S_1(t,y),S_2(t,y)\right)  = 
\iint\,S_1(t,y)\,S_2(t,y)\,dy\,dt\;,
\end{equation}
\begin{equation}
\left(M_1(z,x),M_2(z,x)\right)  = 
\iint\,M_1(z,x)\,M_2(z,x)\,dx\,dz\;.
\end{equation}
\par
The notion of the adjoint operator completely depends on the
arbitrarily chosen definition of the dot product and norm in the model
and data spaces. A simple way to change those definitions is to find
some positive weights $W_M(z,x)$ in the model space and $W_S(t,y)$ in
the data space that define the dot products as follows:
\begin{eqnarray}
\left(S_1(t,y),S_2(t,y)\right) & = &
\iint\,W_S(t,y)\,S_1(t,y)\,S_2(t,y)\,dy\,dt\;,
\label{eqn:wsproduct} \\
\left(M_1(z,x),M_2(z,x)\right) & = &
\iint\,W_M(z,x)\,M_1(z,x)\,M_2(z,x)\,dx\,dz\;.
\label{eqn:wmproduct}
\end{eqnarray}
\par
To formally define the adjoint of a stacking operator, 
let us substitute the definition of the stacking
operator (\ref{eqn:operator}) into the dot product
(\ref{eqn:dottest}), as follows:
\begin{equation}
\left(S(t,y),{\bf A}[M(z,x)]\right) =
\int\iint\,w(x;t,y)\,M(\theta(x;t,y),x)\,S(t,y)\,dx\,dy\,dt\;.
\label{eqn:dot1}
\end{equation}
Assuming that the function $\theta$ is monotone in $t$
\footnote{If this is not
  the case, a different parameterization of the stacking function is
  appropriate \cite[]{antial}}, we can change the integration variable $t$ to
$z=\theta(x;t,y)$ and rewrite equation~(\ref{eqn:dot1}) in the form
\begin{equation}
\left(S(t,y),{\bf A}[M(z,x)]\right) =
\int\iint\,\widetilde{w}(y;z,x)\,M(z,x)\,
S(\widehat{\theta}(y;z,x),x)\,dy\,dx\,dz\;,
\label{eqn:dot2}
\end{equation}
where $\widehat{\theta}$ has the same meaning as in equation
(\ref{eqn:inverse}), and
\begin{equation}
\widetilde{w}(y;z,x) = w(x;\widehat{\theta}(y;z,x),y)\,
\left|\partial \widehat{\theta} \over \partial z\right|\;.
\label{eqn:tildew}
\end{equation}
Comparing equations~(\ref{eqn:dot2}) and (\ref{eqn:dottest}), we conclude that the adjoint
operator ${\bf A^{T}}$ is defined by the equality
\begin{equation}
{\bf A^{T}}[S(t,y)]=
\int \widetilde{w}(y;z,x)\,S(\widehat{\theta}(y;z,x),y)\;dy\;.
\label{eqn:adjoint}
\end{equation}
Thus we have proven that the continuous adjoint of a
stacking operator is another stacking operator. The adjoint operator
has the same summation path as the asymptotic inverse (\ref{eqn:inverse}),
which guarantees the correct reconstruction of the kinematics of the
input wavefield. The amplitude (weighting function) of the adjoint
operator is directly proportional to the forward weighting according
to equation (\ref{eqn:tildew}). The coefficient of proportionality is the
Jacobian of the transformation of the variables $z$ and $t$.
\par
Similar results have been obtained for particular cases of stacking
operators: velocity transform
\cite[]{Thorson.sepphd.39,Jedlicka.sep.61.41}, Kirchhoff
constant-velocity migration \cite[]{Ji.sep.80.499}, and NMO
\cite[]{Crawley.sep.89.207}.  In the appendix, I exemplify an
application of least-squares inversion by reviewing inversion of the
Radon operator and showing that it is precisely equivalent to the
asymptotic result of the previous section.


%%% Local Variables: 
%%% mode: latex
%%% TeX-master: t
%%% TeX-master: t
%%% End: 
