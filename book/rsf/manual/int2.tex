\section{2-D interpolation (int2.c)}
\index{interpolation!2D}




\subsection{{sf\_int2\_init}}\label{sec:sf_int2_init}
Initializes the required variables and allocates the required space for 2D interpolation.

\subsubsection*{Call}
\begin{verbatim}sf_int2_init (coord, o1, o2, d1, d2, n1, n2, interp, nf_in, nd_in);\end{verbatim}

\subsubsection*{Definition}
\begin{verbatim}
void  sf_int2_init (float** coord          /* coordinates [nd][2] */, 
                    float o1, float o2, 
                    float d1, float d2,
                    int   n1, int   n2     /* axes */, 
                    sf_interpolator interp /* interpolation function */, 
                    int nf_in              /* interpolator length */, 
                    int nd_in              /* number of data points */)
/*< initialize >*/
{
   ...
}
\end{verbatim}

\subsubsection*{Input parameters}
\begin{desclist}{\tt }{\quad}[\tt interp]
   \setlength\itemsep{0pt}
   \item[coord]   coordinates (\texttt{float**}).  
   \item[o1]     origin of the first axis (\texttt{float}).  
   \item[o2]     origin of the second axis (\texttt{float}).  
   \item[d1]     sampling of the first axis (\texttt{float}).  
   \item[d2]     sampling of the second axis (\texttt{float}).  
   \item[n1]     length of the first axis (\texttt{float}).  
   \item[n2]     length of the second axis (\texttt{float}).  
   \item[interp] interpolation function (\texttt{sf\_interpolator}).  
   \item[nf\_in] interpolator length (\texttt{int}).  
   \item[nd\_in] number of data points (\texttt{int}).  
\end{desclist}




\subsection{{sf\_int2\_lop}}
Applies the linear operator for 2D interpolation.

\subsubsection*{Call}
\begin{verbatim}sf_int2_lop (adj, add, nm, ny, x, ord);\end{verbatim}

\subsubsection*{Definition}
\begin{verbatim}
void  sf_int2_lop (bool adj, bool add, int nm, int ny, float* x, float* ord)
/*< linear operator >*/
{ 
   ...
}
\end{verbatim}

\subsubsection*{Input parameters}
\begin{desclist}{\tt }{\quad}[\tt add]
   \setlength\itemsep{0pt}
   \item[adj] a parameter to determine whether the output is \texttt{x} or \texttt{ord} (\texttt{bool}).
   \item[add] a parameter to determine whether the input needs to be zeroed (\texttt{bool}).
   \item[nm]  size of \texttt{x} (\texttt{int}).
   \item[ny]  size of \texttt{ord} (\texttt{int}).
   \item[x]   output or operator, depending on whether \texttt{adj} is true or false (\texttt{float*}).
   \item[ord] output or operator, depending on whether \texttt{adj} is true or false (\texttt{float*}).
\end{desclist}




\subsection{{sf\_int2\_close}}
Frees the space allocated for 2D interpolation by \hyperref[sec:sf_int2_init]{\texttt{sf\_int2\_init}}.

\subsubsection*{Call}
\begin{verbatim}sf_int2_close();\end{verbatim}

\subsubsection*{Definition}
\begin{verbatim}
void sf_int2_close (void)
/*< free allocated storage >*/
{
   ...
}
\end{verbatim}

