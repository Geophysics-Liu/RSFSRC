\title{Revisiting Velocity Analysis Demo from SU \\ with RSF and SCons}

\email{sergey.fomel@beg.utexas.edu}

\author{Sergey Fomel}

\righthead{Velocity Analysis Demo}

\lefthead{Fomel}

\maketitle

\begin{abstract}
In this guide, we will reproduce the velocity analysis demo from
Seismic Unix in order to demonstrate several RSF programs and 
some general features of working with RSF and SCons:
\begin{itemize}
  \item Operating with regular and irregular files
  \item Using different plotting programs
  \item Combining figures
  \item Using functions and loops in SCons
\end{itemize}
and more.
\end{abstract}

\section{Introduction}

Seismic Unix (SU) is a package maintained by John Stockwell at the
Center for Wave Phenomenon at the Colorado School of Mines
\cite[]{TLE16-07-10451049,su}\footnote{SU is available at
\url{http://timna.mines.edu/cwpcodes/}.}. In this guide, we will
reproduce the demo found in
\texttt{/src/demos/Velocity\_Analysis/Traditional} directory in the SU
distribution. Step by step instructions will familiarize you with RSF
tools and working with Scons.

\definecolor{frame}{rgb}{0.905,0.905,0.905}
\lstset{language=Python,backgroundcolor=\color{frame},showstringspaces=false,numbers=left,numberstyle=\tiny}

\inputdir{rsf}

\section{Setting up}

A complete listing of the \texttt{SConstruct} file used in this guide
is given in the appendix. Please take a quick look at it before
proceeding further to get an idea of its general structure. We will
examine the file in more detail in the next sections. If you have a
working RSF installation, you should be able to reproduce all results
from this paper by simply copying the \texttt{SConstruct} file to any
directory on your system and running \texttt{scons} there. The file is
available on the web at...

Notice three basic commands reccuring in the \texttt{SConstruct} file:
\begin{description}
\item[\texttt{Flow(target,source,command)}] defines how the
  \texttt{target} file is build from the \texttt{source} file using the
  \texttt{command}. Note that the \texttt{Flow} command does not
  execute anything. It just sets a rule for the file creation. The
  rule is executed when you run \texttt{scons target.rsf} or build
  another target that has \texttt{target.rsf} in its dependencies.
\item[\texttt{Plot(target,source,command)}] is a version of
  \texttt{Flow} used for creating figures. The output of \texttt{Plot}
  is a vplot figure \texttt{target.vpl} rather than an RSF file.
  \texttt{Plot} is also used to combine several figures into one.
\item[\texttt{Result(target,source,command)}] is similar to
  \texttt{Plot}, only the output figure is considered a repreoducible
  result and placed in a special directory (\texttt{./Fig} by
  default). Executing the rule specified in \texttt{Result} typically
  results in a file \texttt{Fig/target.vpl}
\end{description}

Several default rules are set up for convenience of working with
\texttt{SCons}.
\begin{description}
\item[scons] blah
\item[scons view] blah
\item[scons lock] blah
\item[scons test] blah
\end{description}

\section{Synthetic data modeling}

\plot{modl}{width=\textwidth}{rsf/modl} \clearpage
\plot{modl2}{width=\textwidth}{rsf/modl2} \clearpage
\plot{data}{width=\textwidth}{rsf/data} \clearpage

\section{Processing individual common midpoint gathers}

\plot{cdp1500}{width=\textwidth}{rsf/cdp1500} \clearpage
\plot{cdp2000}{width=\textwidth}{rsf/cdp2000} \clearpage
\plot{cdp2500}{width=\textwidth}{rsf/cdp2500} \clearpage
\plot{cdp3000}{width=\textwidth}{rsf/cdp3000} \clearpage
\plot{cdp3500}{width=\textwidth}{rsf/cdp3500} \clearpage

\section{Processing the complete dataset}

\plot{pick}{width=\textwidth}{rsf/pick} \clearpage
\plot{stack}{width=\textwidth}{rsf/stack} \clearpage

\section{Conclusions}

This tour is not designed as a comprehensive manual. It simply gives a
glimpse into working in a reproducible research environment with RSF
and SCons. The reader is encouraged to experiment with the
\texttt{SConstruct} file attached to this tour and included in the
Appendix. For other documentation on RSF, please see
\begin{itemize}
\item \href{http://www.reproducibility.org/RSF/book/rsf/rsf/tour_html/}{Introduction to RSF}  
\item  \href{http://www.reproducibility.org/RSF/book/rsf/rsf/install_html/}{Installation instructions}
\item \href{http://www.reproducibility.org/RSF/}{Self-documentation reference for RSF programs}
\item A \href{http://www.reproducibility.org/RSF/book/rsf/rsf/prog_html/}{guide to RSF programs}
\item A \href{http://www.reproducibility.org/RSF/book/rsf/rsf/format_html/}
  {guide to RSF file format}
\item A \href{http://www.reproducibility.org/RSF/book/rsf/rsf/api_html/}{guide to
    RSF programming interface}
\item A \href{http://www.reproducibility.org/RSF/book/rsf/rsf/demo_html/}{guide to programming with RSF}
\item A \href{http://www.reproducibility.org/RSF/book/rsf/rsf/tour_html/}{tour of RSF software}
\item A
  \href{http://www.reproducibility.org/RSF/book/rsf/scons/paper_html/}{guide
    to SCons interface for reproducible computations}
\end{itemize}

\bibliographystyle{seg}
\bibliography{intro,SEG}

\append{SConstruct file}

Here is a complete listing of the \texttt{SConstruct} file used in this
example.

\lstinputlisting[frame=single]{rsf/SConstruct}
