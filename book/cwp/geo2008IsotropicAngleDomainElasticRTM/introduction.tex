% ------------------------------------------------------------
\section{Introduction}
% ------------------------------------------------------------

Seismic processing is usually based on acoustic wave equations, which
assume that the Earth represents a liquid that propagates only
compressional waves. Although useful in practice, this assumption is
not theoretically valid.  Earth materials allow for both \geosout{primary}
\geouline{compressional} and shear wave propagation in the
subsurface. Shear waves, either generated at the source or converted
from compressional waves at various interfaces in the subsurface, are
\geosout{recorded} \geouline{detected} by multicomponent receivers.  Shear
waves are usually stronger at large incidence and reflection angles,
often corresponding to large offsets. However, for complex geological
structures near the surface, shear waves can be quite significant even
at small offsets.  Conventional single-component imaging ignores shear
wave modes, which often leads to incorrect characterization of
wave-propagation, incomplete illumination of the subsurface and poor
amplitude characterization\geosout{, etc}.

Even when multicomponent data are used for imaging, they are usually
not processed with specifically designed procedures. Instead, those
data are processed with ad-hoc procedures borrowed from acoustic wave
equation imaging algorithms. \geouline{For isotropic media,} \geosout{A}
\geouline{a} typical assumption is that the recorded vertical and
\geouline{in plane} horizontal components are good approximations for the
P- and S-wave modes, respectively, which can be imaged
independently. This assumption is not always correct, leading to
errors and noise in the images, since P- and S-wave modes are normally
mixed on all recorded components. Also, since P and S modes are mixed
on all components, true-amplitude imaging is questionable no matter
how accurate the wavefield reconstruction and imaging condition are.

Multicomponent imaging has long been an active research area for
exploration geophysicists.  Techniques proposed in the literature
perform imaging \geosout{in the time domain} \geouline{by using time
  extrapolation}, e.g. by Kirchhoff migration
\cite[]{kuo:1223,hokstad:861} and reverse-time
migration~\cite[]{whitmore.thesis,chang:67,chang:597} adapted for
multicomponent data. The reason for working in the time domain, as
opposed to the depth domain, is that the coupling of displacements in
different directions in elastic wave equations makes it difficult to
derive a dispersion relation that can be used to extrapolate
wavefields in depth \cite[]{Clayton.sep.20.73,Clayton.thesis}.


Early attempts at multicomponent imaging used the Kirchhoff framework
\geouline{and involve wave-mode separation on the surface prior to
  wave-equation imaging} \cite[]{WapenaarEtAl1987,WapenaarHaime1990}.
\cite{GEO49-08-12231238} perform\geosout{s} shot-profile elastic Kirchhoff
migration\geouline{,} and \cite{hokstad:861} perform\geouline{s} survey-sinking
elastic Kirchhoff migration. Although these techniques represent
different migration procedures, they compute travel-times for both PP
and PS reflections, and sum data along these travel time
trajectories. This approach is equivalent to distinguishing between PP
reflection and PS reflections, and applying acoustic Kirchhoff
migration for each mode separately. When geology is complex, the
elastic Kirchhoff migration technique suffers from drawbacks similar
to those of acoustic Kirchhoff migration because ray theory breaks
down \cite[]{GEO66-05-16221640}.

There are two main difficulties with independently imaging P and S
wave modes separated on the surface.  The first is that conventional
elastic migration techniques either consider vertical and horizontal
components of recorded data as P and S modes, which is not always
accurate, or separate these wave modes on the recording surface using
approximations, e.g.  polarization \cite[]{pestana:1308} or elastic
potentials \cite[]{etgen:972,GEO62-02-05980613} \geouline{or wavefield
extrapolation in the vicinity of the acquisition surface}
\cite[]{WapenaarEtAl1990,AmundsenReitan1995}. Other elastic reverse
time migration techniques do not separate wave modes on the surface
and reconstruct vector fields, but use imaging conditions based on ray
tracing \cite[]{chang:67,chang:597} that are not always robust in
complex geology. The second difficulty is that images produced
independently from P and S modes are hard to interpret together, since
often they do not line-up consistently, thus requiring image post
processing, e.g. by manual or automatic registration of the images
~\cite[]{Gaiser1996,fomel:781,nickel:869}.

We advocate an alternative procedure for imaging elastic wavefield
data. Instead of separating wavefields into scalar wave modes on the
acquisition surface followed by scalar imaging of each mode
independently, we use the entire vector wavefields for wavefield
reconstruction and imaging.  The vector wavefields are reconstructed
using the multicomponent vector data as boundary conditions for a
numerical solution to the elastic wave equation. The key component of
such a migration procedure is the imaging condition which evaluates
the match between wavefields reconstructed from the source and
receiver. For vector wavefields, a simple component-by-component
cross-correlation between the two wavefields leads to artifacts caused
by crosstalk between the unseparated wave modes, i.e. all P and S
modes from the source wavefield correlate with all P and S modes from
the receiver wavefield. This problem can be alleviated by using
separated elastic wavefields, with the imaging condition implemented
as cross-correlation of wave modes instead of cross-correlation of the
Cartesian components of the wavefield. This approach leads to images
that are cleaner and easier to interpret since they represent
reflections of single wave modes at interfaces of physical properties.

As for imaging with acoustic wavefields, the elastic imaging condition
can be formulated conventionally (cross-correlation with zero lag in
space and time), as well as extended to non-zero space lags. The
elastic images produced by extended imaging condition can be used for
angle decomposition of PP and PS reflectivity. Angle gathers have many
applications, including migration velocity analysis (MVA) and
amplitude versus angle (AVA) analysis.

The advantage of imaging with multicomponent seismic data is that the
physics of wave propagation is better represented, and resulting
seismic images more accurately characterize the subsurface.
Multicomponent images have many applications. For example they can be
used to provide reflection images where the P-wave reflectivity is
small, image through gas clouds where the P-wave signal is attenuated,
\geosout{detect fractures through shear-wave splitting,} validate bright
spot reflections and provide parameter estimation for this media,
\geosout{and} Poisson's ratio estimates, \geouline{and detect fractures
through shear-wave splitting for anisotropic media}
\cite[]{li:2056,zhu:1283,knapp:776,gaiser:974,stewart:40,simmons:1227}.
Assuming no attenuation in the subsurface, converted wave images also
have higher resolution than pure-mode images in shallow part of
sections, because S-waves have shorter wavelengths than P-waves.
Modeling and migrating multicomponent data with elastic migration
algorithms enables us to make full use of information provided by
elastic data and correctly position geologic structures.

%% 
 % what we        discuss in this paper 
 % what we do not discuss in this paper
 %%

\geouline{This paper presents a method for angle-domain imaging of
elastic wavefield data using reverse-time migration (RTM). In order to
limit the scope of our paper, we ignore several practical issues
related to data acquisition and pre-processing for wave-equation
migration. For example, our methodology ignores the presence of
surface waves, e.g. Rayleigh and Love waves, the relatively poor
spatial sampling when imaging with multicomponent elastic data,
e.g. for OBC acquisition, the presence of anisotropy in the subsurface
and all amplitude considerations related to the directionality of the
seismic source. All these issues are important for elastic imaging and
need to be part of a practical data processing application. We
restrict in this paper our attention to the problem of wave-mode
separation after wavefield extrapolation and angle-decomposition after
the imaging condition. These issues are addressed in more detail in a
later section of the paper.}

We begin by summarizing wavefield imaging methodology, focusing on
reverse-time migration for wavefield multicomponent migration. Then,
we describe different options for wavefield multicomponent imaging
conditions, e.g. based on vector displacements and vector potentials.
Finally, we describe the application of extended imaging
condition\geouline{s} to multicomponent data and corresponding angle
decomposition. We illustrate the wavefield imaging techniques using
data simulated from the Marmousi II model \cite[]{martin:1979}.
