% ------------------------------------------------------------
\section{Extended elastic imaging conditions}
% ------------------------------------------------------------

The conventional imaging condition from \req{CIC} discussed in the
preceding section uses zero \geouline{space- and time-}lags of the
cross-correlation between the source and receiver wavefields. This
imaging condition represents a special case of a more general form of
\geosout{imaging condition, sometimes referred-to as }an extended imaging
condition \cite[]{SavaFomel.geo.tsic}
%
\beq \label{eqn:XIC}
\IM{}\ofxlt = \int \US{}\ofxlm \UR{}\ofxlp dt \;,
\eeq
%
where $\hh=\{\lambda_x,\lambda_y,\lambda_z\}$ and $\tt$ stand for
cross-correlation lags in space and time, respectively. The imaging
condition from \req{CIC} is equivalent to the extended imaging condition
from \req{XIC} for $\hh=\mathbf{0}$ and $\tt=0$.

The extended imaging condition has two main uses.  First, it
characterizes wavefield reconstruction errors, since for incorrectly
reconstructed wavefields, the cross-correlation energy does not focus
completely at zero lags in space and time. Sources of wavefield
reconstruction errors include inaccurate numeric solutions to the
wave-equation, inaccurate \geosout{(velocity)} models used for
wavefield reconstruction, inadequate wavefield sampling on the
acquisition surface, \geouline{and} uneven illumination of the
subsurface \geosout{, etc}. Typically, all these causes of inaccurate
wavefield reconstruction occur simultaneously and it is difficult to
separate them after imaging. Second, assuming accurate wavefield
reconstruction, the extended imaging condition can be used for angle
decomposition. This leads to representations of reflectivity
\geouline{as a} function of angles of incidence and reflection at all
points in the imaged volume \cite[]{SavaFomel.geo.ang}. Here, we
assume that wavefield reconstruction is accurate and concentrate on
further extensions of the imaging condition, such as angle
decomposition.


% ------------------------------------------------------------
\inputdir{XFig}
% ------------------------------------------------------------

\multiplot[!htpb]{1}{cwang}{width=\textwidth}{Local wave vectors of the
converted wave at a common image point location in 3D. The plot shows
the conversion in the reflection plane in 2D. $\pp_\ss$, $\pp_\rr$,
$\pp_\xx$ and $\pp_\hh$ are ray parameter vectors for the source ray,
receiver ray, and combinations of the two. The length of the incidence
and reflection wave vectors are inversely proportional to the
incidence and reflection wave velocity, respectively. Vector ${\bf n}$
is the normal of the reflector. By definition,
\geosout{$\pp_\xx=\pp_\rr+\pp_\ss$ and $\pp_\hh=\pp_\rr-\pp_\ss$}
\geouline{$\pp_\xx=\pp_\rr-\pp_\ss$ and $\pp_\hh=\pp_\rr+\pp_\ss$}
.}
% ------------------------------------------------------------
\subsection{Imaging with vector displacements}

For \geosout{the case of} imaging with vector wavefields, the extended
imaging condition from \req{XIC} can be applied directly to the various
components of the reconstructed source and receiver wavefields,
\geosout{similarly} \geouline{similar} to the conventional imaging procedure
described in the preceding section. Therefore, an extended image
constructed from vector displacement wavefields is
%
\beq \label{eqn:XICa}
\IM{ij}\ofxlt = \int \US{i}\ofxlm \UR{j}\ofxlp dt \;,
\eeq
%
where the quantities $u_{si}$ and $u_{rj}$ stand for the Cartesian
components ${x,y,z}$ of the vector source and receiver wavefields, and
$\hh$ and $\tt$ stand for cross-correlation lags in space and time,
respectively. This imaging condition suffers from the same drawbacks
described for the similar conventional imaging condition applied to
the Cartesian components of the reconstructed wavefields,
i.e. crosstalk between the unseparated wave modes\geosout{, etc}.

% ------------------------------------------------------------
\subsection{Imaging with scalar and vector potentials}

An extended imaging condition can also be designed for elastic
wavefields decomposed in scalar and vector potentials, \geosout{similarly}
\geouline{similar} to the conventional imaging procedure described in the
preceding section. Therefore, an extended image constructed from
scalar and vector potentials is
%
\beq \label{eqn:XICe}
\IM{ij}\ofxlt = \int \MS{i}\ofxlm \MR{j}\ofxlp dt \;,
\eeq
%
where the quantities \geosout{$alpha_{si}$ and $alpha_{rj}$}
\geouline{$\alpha_{si}$ and $\alpha_{rj}$} stand for the various wave
modes $\alpha=\{P, S\}$ of the source and receiver wavefields, and
$\hh$ and $\tt$ stand for cross-correlation lags in space and time,
respectively.  
