

\section{Discussion}
% present principle, relationships and generalizatios shown by resuts
The separation of P and S wave-modes is based on the projection of
elastic wavefields onto their respective polarization vectors. For VTI
media, P and S mode polarization vectors can be conveniently obtained
by solving the Christoffel equation. The Christoffel equation is a
plane-wave solution to the elastic wave equation. Since the
displacements, velocity and acceleration field have the same form of
elastic wave equation, the separation algorithm applies to all these
wavefields.  The P and SV mode separation can be extended to TTI
(transverse isotropy with a tilted symmetry axis) media by solving a
TTI Christoffel matrix, and obtain TTI separators. Physically, the TTI
media is just a rotation of VTI media.

% point out any exceptions or any lack of correlations and define
% unsettled points
In TI media, SV and SH waves are uncoupled most of the time, where SH
wave is polarized out of plane.  One only needs to decompose P and SV
modes in the vertical plane. The plane wave solution is sufficient for
most TI media, except for a special case where there exists a
singularity point at an oblique propagation angle in the vertical
plane (a line singularity in 3D), at which angle SV and SH wave
velocities coincide. At this point, the SV wave polarization is not
uniquely defined by Christoffel equation. S waves at the singularity
are polarized in a plane orthogonal to the P wave polarization
vector. However, this is not a problem since we define SV waves
polarized in vertical planes only, therefore I remove the singularity
by using the cylindrical coordinates. This situation is similar to S
wave-mode coupling in orthorhombic media, where there is at least one
singularity in a quadrant. {However, as pointed out
by \cite{GEO55-07-09140919}, the singularity in orthorhombic media is
a global property of the media and cannot be removed, therefore the
separation using polarization vectors in 3D orthorhombic media is not
straightforward.}





%sensitivity
The anisotropic derivative operators depend on the anisotropic medium
parameters. In \rFg{separate5-pA1,pA2,pA3}, I show how sensitive the
separation is to the medium parameters. One elastic wavefield snapshot
is shown in \rFg{separate2-uA} for a VTI medium with
$V_{P0}/V_{S0}=2$ and $\epsilon=0.25$, $\delta=-0.29$. I try to separate
the P and SV modes with (a) $\epsilon=0.4$, $\delta=-0.1$, (b)
$\epsilon=0$, $\delta=-0.3$ and (c) $\epsilon=0$, $\delta=0$ . The
separation shows that parameters (a) have good separation, showing the
difference in $\epsilon$ and $\delta$ is important. The worst case
scenario is shown by parameters (c), where isotropy is assumed for
this VTI medium.

\inputdir{separate5}
\multiplot{3}{pA1,pA2,pA3}{width=.65\textwidth}
{P and SV wave mode separation for a snapshot shown in
{\rFg{separate2-uA}}. The true medium parameters are $\epsilon=0.25$,
$\delta=-0.29$. The separation assumes medium parameters of
(a)$\epsilon=0.4$, $\delta=-0.1$, (b) $\epsilon=0$, $\delta=-0.3$, and
(c)$\epsilon=0$, $\delta=0$. Hard clipping was applied to show the
weak events. {The plot shows that different estimate of anisotropy
parameters has influence on the the wave mode separation.}}
