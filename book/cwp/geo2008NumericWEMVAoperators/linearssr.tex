\subsection{Mixed-domain operators}

For the case of the phase-shift operation in media with lateral
slowness variation, the mixed-domain solution involves forward and
inverse Fourier transforms (denoted fFT and iFT in our algorithms)
which can be implemented efficiently using standard Fast Fourier
Transform algorithms. The numeric implementation is summarized in the
following table:
% ------------------------------------------------------------
\psop
% ------------------------------------------------------------
%% 
 % explain algorithm here
 %%

In this chart, $\kzwk$ denotes the $\ww-\kk$ component of the depth
wavenumber and $\kzwx$ denotes the $\ww-\xx$ component of the depth
wavenumber. An example of mixed-domain implementation is the
Split-Step Fourier (SSF) method, where $\kzwk$ represents the SSR
equation computed with a constant reference slowness $\tilde{s}$, and
$\kzwx=\ww\lp s-\tilde{s} \rp$ represents a space-domain correction
\cite[]{GEO55-04-04100421}.

Based on the equation \ren{SSR-SR}, the derivative of the depth
wavenumber relative to slowness is
%
\beq \label{eqn:DKZDS-sroot}
\dkzds = \frac{\ww}{\SSX{\ws_0\ofm}{\km}} \;.
\eeq
%
The numeric implementation of the pseudo-differential equation
\ren{DKZDS-sroot} is as complicated in media with lateral slowness
variation as its phase-shift counterpart (\req{SSR-ZO}). However, we
can construct efficient and robust numeric implementations using
similar approximations as the ones employed for the phase-shift
relation, e.g. mixed-domain numeric implementation.

The linearized scattering operator can also be implemented in a
mixed-domain by expanding the square-root from relation
\ren{DKZDS-sroot} using a Taylor series expansion
%
\beq \label{eqn:DKZDS-taylor}
\dkzds \approx \ww \lp \SST{\ws_0\ofm}{\km} \rp \;,
\eeq
%
where $c_j$ are binomial coefficients of the Taylor series.

Therefore, the wavefield perturbation at depth $z$ caused by a
slowness perturbation at depth $z$ under the influence of the
background wavefield at the same depth $z$ (forward scattering
operator \ren{ZOFSOP}) can be written as
\beq \label{eqn:FSOPnumeric}
\dUU\ofm \approx \pm i\ww\dz \lp \SST{\ws_0\ofm}{\km} \rp \UU\ofm \ds\ofm \;.
\eeq

Similarly, the slowness perturbation at depth $z$ caused by a
wavefield perturbation at depth $z$ under the influence of the
background wavefield at the same depth $z$ (adjoint scattering
operator \ren{ZOASOP}) can be written as
\bea \label{eqn:ASOPnumeric}
\ds\ofm \approx \mp i\ww\dz \lp \SST{\ws_0\ofm}{\km} \rp \CONJ{\UU\ofm} \dUU\ofm \;.
\eea

The mixed-domain implementation of the forward and adjoint scattering
operators \ren{FSOPnumeric} and \ren{ASOPnumeric}, is summarized on
the following tables:
% ------------------------------------------------------------
%\newpage
\vfill \scop \vfill \acop \vfill
% ------------------------------------------------------------
%% 
 % explain algorithms here
 %%
