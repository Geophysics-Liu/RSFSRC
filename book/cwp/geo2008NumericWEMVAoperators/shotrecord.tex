% ------------------------------------------------------------
\subsection{Shot-record migration and velocity analysis}

Wavefield reconstruction for multi-offset migration based on the
one-way wave-equation under the shot-record framework is performed by
separate recursive extrapolation of the source and receiver
wavefields, $\US$ and $\UR$. The wavefield extrapolation progresses
forward in time (causal) for the source wavefield and backward in time
(anti-causal) for the receiver wavefield:
\bea \label{eqn:PHSs-SR}
\US_{z+\dz}\ofm &=& \PS{+}\US_z\ofm
\\   \label{eqn:PHSr-SR}
\UR_{z+\dz}\ofm &=& \PS{-}\UR_z\ofm
\eea
In \reqs{PHSs-SR}-\ren{PHSr-SR}, $\US_z\ofm$ and $\UR_z\ofm$ represent
the source and receiver acoustic wavefield for a given frequency $\ww$
at all positions in space $\mm$ at depth $z$, and $\US_{z+\dz}\ofm$
and $\UR_{z+\dz}\ofm$ represent the same wavefields extrapolated to
depth $z+\dz$. The phase shift operation uses the depth wavenumber
$\kz$ which is defined by the single square-root (SSR) equation
\beq \label{eqn:SSR-SR}
\kz = \SSR{\ws\ofm}{\km}
\eeq
%
The image is obtained from the extrapolated wavefields by selection of
the zero cross-correlation lags in space of time between the source
and receiver wavefields, an operation which is usually implemented as
summation over frequencies:
%
\beq \label{eqn:IMC-SR}
\RR_z\ofm = \sum_\ww \CONJ{\US_z\ofmw} \UR_z\ofmw \;.
\eeq
An alternative imaging condition \cite[]{SavaFomel.geo.tsic} preserves
the space and time cross-correlation lags in the image.


Linearizing the depth wavenumber given by the equation \ren{SSR-SR}
relative to the background slowness $s_0\ofm$ similarly to the case
case of zero-offset migration, we can reconstruct the acoustic
wavefields in the background model using a phase-shift operation
%
\bea
\US_{z+\dz}\ofm &=& \PSo{+}\US_z\ofm \;,
\\
\UR_{z+\dz}\ofm &=& \PSo{-}\UR_z\ofm \;,
\eea
%
which define the causal $\PSop{+}{SRM}{{s_0}_z\ofm,\UU_z\ofm}$ and the
anti-causal $\PSop{-}{SRM}{{s_0}_z\ofm,\UU_z\ofm}$ wavefield
extrapolation operators for shot-record migration constructed using
the background slowness $s_0\ofm$ and producing the wavefields
$\US_{z+\dz}\ofm$ and $\UR_{z+\dz}\ofm$ at depth $z+\dz$ from the
wavefields $\US_z\ofm$ and $\UR_z\ofm$ at depth $z$, respectively. A
typical implementation of shot-record wave-equation migration follows
the algorithm:
% ------------------------------------------------------------
%\newpage
\srmig
% ------------------------------------------------------------
This algorithm is similar to the one used for zero-offset or survey
sinking migration, except that the source and receiver wavefields are
reconstructed separately using wavefield extrapolation. Unlike the
zero-offset extrapolation operator, the shot-record extrapolation
operator uses the background slowness $s_0$ since the operation
involves sinking of the source and receiver wavefields from the
surface toward the image positions. Wavefield extrapolation is usually
implemented in a mixed domain (space-wavenumber), as briefly
summarized in Appendix A.

Similarly to the derivation of the wavefield perturbation in the
zero-offset migration case, we can write the linearized wavefield
perturbation for shot-record migration as
%
\bea
\dUS\ofm &\approx& +i \dkzds \dz \;                   \US\ofm \ds\ofm
\nonumber \\  \label{eqn:SRFSOPs}
         &\approx& +i\dz \SQREXP{\ww\US\ofm \ds\ofm}{\ws_0\ofm}{\km} \;,
\eea
and
\bea
\dUR\ofm &\approx& -i \dkzds \dz \;                   \UR\ofm \ds\ofm
\nonumber \\  \label{eqn:SRFSOPr}
         &\approx& -i\dz \SQREXP{\ww\UR\ofm \ds\ofm}{\ws_0\ofm}{\km} \;.
\eea
%
\rEqs{SRFSOPs}-\ren{SRFSOPr} define the forward scattering operators 
$\FSop{\pm}{SRM}{\UU\ofm,s_0\ofm,\ds\ofm}$ producing the scattered
wavefields $\dUU\ofm$ from the slowness perturbation $\ds\ofm$, based
on the background slowness $s_0\ofm$ and background wavefield
$\UU\ofm$. In this case, the symbol $\UU$ stands for either $\US$ or
$\UR$, given the appropriate choice of sign in the forward scattering
operator. The image perturbation at depth $z$ is obtained from the
source and receiver scattered wavefields using the relation
\beq
\dR\ofm = \sum_\ww \lp \CONJ{\US\ofmw} \dUR\ofmw + 
                       \CONJ{\dUS\ofmw} \UR\ofmw \rp \;,
\eeq
which corresponds to the frequency-domain zero-lag cross-correlation
of the source and receiver wavefields required by the imaging
condition.

Given an image perturbation $\dR$, we can construct the scattered
source and receiver wavefields by the adjoint of the imaging condition
\bea
\dUS\ofm = \UR\ofm \CONJ{\dR\ofm}\;,
\\
\dUR\ofm = \US\ofm       \dR\ofm \;,
\eea
for every frequency $\ww$. Then, the slowness perturbations due to the
source and receiver wavefields at depth $z$ under the influence of the
background source and receiver wavefields at the same depth $z$ can be
written as
%
\bea
\ds_s\ofm &\approx& -i \dkzds \dz \;                    \CONJ{\US\ofm} \dUS\ofm
\nonumber \\   \label{eqn:SRASOPs}
          &\approx& -i\dz \SQREXP{\ww\CONJ{\US\ofm} \dUS\ofm}{\ws_0\ofm}{\km} \;,
\eea
and
\bea 
\ds_r\ofm &\approx& -i \dkzds \dz \;                    \CONJ{\UR\ofm} \dUR\ofm 
\nonumber \\  \label{eqn:SRASOPr}
          &\approx& -i\dz \SQREXP{\ww\CONJ{\UR\ofm} \dUR\ofm}{\ws_0\ofm}{\km} \;.
\eea
%
\rEqs{SRASOPs}-\ren{SRASOPr} define the adjoint scattering operators
$\ASop{\pm}{SRM}{\UU\ofm,s_0\ofm,\dUU\ofm}$, producing the slowness
perturbation $\ds\ofm$ from the scattered wavefield $\dUU\ofm$, based
on the background slowness $s_0\ofm$ and background wavefield
$\UU\ofm$. In this case, $\UU$ stands for either $\US$ or $\UR$, given
the appropriate choice of sign in the adjoint scattering operator. A
typical implementation of shot-record forward and adjoint scattering
follows the algorithms:
% ------------------------------------------------------------
%\newpage
\srfor 
%\newpage
\sradj
% ------------------------------------------------------------
These algorithms are similar to the one used for zero-offset or survey
sinking migration, except that the source and receiver wavefields are
reconstructed separately using wavefield extrapolation. Unlike the
zero-offset scattering operators, the shot-record scattering operators
use the background slowness $s_0$ since the operation involves sinking
of the source and receiver wavefields from the surface toward the
image positions. 
%
Both forward and adjoint scattering algorithms require the background
wavefields, $\US\ofm$ and $\UR\ofm$, to be precomputed at all depth
levels.
% 
Scattering and wavefield extrapolation are implemented in the mixed
space-wavenumber domain, as briefly explained in Appendix A.
