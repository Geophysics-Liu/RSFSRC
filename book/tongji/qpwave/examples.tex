\section{EXAMPLES}
\subsection{1. Simulating propagation of separated wave modes}

%%%%%%%%%%%%%%%%%%%%%%%%%%%%%%%%%%%%%%%%%%%%%%%%%%%%%%%%%%%%%%%%%%%%%%%%%%%%%%%%%%%%%%%%%%%%%%%%%%%%%
\subsubsection{1.1 Homogeneous VTI model}
\inputdir{homovti.eta0.05}

For comparison, we first appply the original anisotropic elastic wave equation
 to synthesize wavefields in a homogeneous VTI medium with weak anisotropy, in which
$v_{p0}=3000 m/s$, $v_{s0}=1500 m/s$, $\epsilon=0.1$, and $\delta=0.05$. 
Figure 4a and 4b display the horizontal and vertical components of the displacement wavefields at 0.3 s.
 Then we try to simulate propagation of separated wave modes using the pseudo-pure-mode qP-wave equation 
(equation~\ref{eq:pseudoVTIxy} in its 2D form).
 Figure 4c and 4d display the two components of the pseudo-pure-mode qP-wave fields, and Figure 4e
 displays their summation, i.e., the pseudo-pure-mode scalar qP-wave fields with weak residual qSV-wave energy. 
 Compared with the theoretical wavefront curves (see Figure 4f) calculated
on the base of group velocities
 and angles, pseudo-pure-mode scalar qP-wave fields have correct kinematics for both qP- and qSV-waves.
We finally remove residual qSV-waves and get completely separated scalar qP-wave fields by 
 applying the filtering to correct the projection deviation (Figure 4g).

\multiplot{7}{Elasticx,Elasticz,PseudoPurePx,PseudoPurePz,PseudoPureP,WF,PseudoPureSepP}{width=0.2\textwidth}
{
Synthesized wavefields in a VTI medium with weak anisotropy: (a) x- and
(b) z-components synthesized by original elastic wave equation; (c) x- and
 (d) z-components synthesized by pseudo-pure-mode qP-wave equation; (e) pseudo-pure-mode scalar qP-wave fields; 
(f) kinematics of qP- and qSV-waves; and (g) separated scalar qP-wave fields.
}

%%%%%%%%%%%%%%%%%%%%%%%%%%%%%%%%%%%%%%%%%%%%%%%%%%%%%%%%%%%%%%%%%%%%%%%%%%%%%%%%%%%%%%%%%%%%%%%%%%%%%
\inputdir{homovti.eta0.5}

Then we consider wavefield modeling in a homogeneous VTI medium with strong anisotropy,
 in which $v_{p0}=3000 m/s$, $v_{s0}=1500 m/s$, $\epsilon=0.25$, and $\delta=-0.25$.
 Figure 5 displays the wavefield snapshots at 0.3 s synthesized by using original elastic wave equation
and pseudo-pure-mode qP-wave equation respectively. Note that the pseudo-pure-mode qP-wave fields still accurately
represent the qP- and qSV-waves' kinematics. Although the residual qSV-wave energy becomes stronger when
the strength of anisotropy increases, the filtering step still removes these residual qSV-waves effectively.

\multiplot{7}{Elasticx,Elasticz,PseudoPurePx,PseudoPurePz,PseudoPureP,WF,PseudoPureSepP}{width=0.2\textwidth}
{
Synthesized wavefields in a VTI medium with strong anisotropy: (a) x- and
(b) z-components synthesized by original elastic wave equation; (c) x- and
 (d) z-components synthesized by pseudo-pure-mode qP-wave equation; (e) pseudo-pure-mode scalar qP-wave fields; 
(f) kinematics of qP- and qSV-waves; and (g) separated scalar qP-wave fields.
}

%%%%%%%%%%%%%%%%%%%%%%%%%%%%%%%%%%%%%%%%%%%%%%%%%%%%%%%%%%%%%%%%%%%%%%%%%%%%%%%%%%%%%%%%%%%%%%%%%%%%%
\subsubsection{1.2 Two-layer TI model}
\inputdir{twolayer2dti}

This example demonstrates the approach on a two-layer TI model, in which the first layer is a very
strong VTI medium with $v_{p0}=2500 m/s$, $v_{s0}=1200 m/s$, $\epsilon=0.25$, and $\delta=-0.25$, 
and the second layer is a TTI medium with $v_{p0}=3600 m/s$, $v_{s0}=1800 m/s$, 
$\epsilon=0.2$, $\delta=0.1$, and $\theta=30^{\circ}$. The horizontal interface between the two layers 
is positioned at a depth of 1.167 km.
 Figure 6a and 6d display the horizontal and vertical components of the displacement wavefields at 0.3 s.
 Using the pseudo-pure-mode qP-wave equation, we simulate equivalent wavefields on the same model.
Figure 6b and 6e display the two components of the pseudo-pure-mode qP-wave fields at the same time step.
Figure 6c and 6f display pseudo-pure-mode scalar qP-wave fields and separated qP-wave fields respectively.
Obviously, residual qSV-waves (including transmmited, reflected 
and converted qSV-waves) are effectively removed, and all transmitted, reflected as well as converted
qP-waves are accurately separated after the projection deviation correction.

\multiplot{6}{ElasticxInterf,PseudoPurePxInterf,PseudoPurePInterf,ElasticzInterf,PseudoPurePzInterf,PseudoPureSepPInterf}{width=0.25\textwidth}
{
Synthesized wavefields on a two-layer TI model with strong anisotropy and a tilted symmetry axis: (a) x- and 
(d) z-components synthesized by original elastic wave equation; (b) x- and
 (e) z-components synthesized by pseudo-pure-mode qP-wave equation; 
 (c) pseudo-pure-mode scalar qP-wave fields; (f) separated scalar qP-wave fields.
}


%%%%%%%%%%%%%%%%%%%%%%%%%%%%%%%%%%%%%%%%%%%%%%%%%%%%%%%%%%%%%%%%%%%%%%%%%%%%%%%%%%%%%%%%%%%%%%%%%%%%%
\subsubsection{1.3 BP 2007 TTI model}
\inputdir{bptti2007}

Next we test the approach of simulating propagation of the separated qP-wave mode
in a complex TTI model. Figure 7 shows parameters for part of
the BP 2D TTI model. The space grid size is 12.5 m and the time step is 1 ms for 
high-order finite-difference operators. Here the vertical velocities for the qSV-wave are set as
 half of the qP-wave velocities. 
Figure 8 displays snapshots of wavefield components at the time of 1.4s
synthesized by using original elastic wave equation and pseudo-pure-mode qP-wave equation.
The two pictures on
the right side represent the scalar pseudo-pure-mode qP-wave and the separated qP-wave fileds, respectively.
The correction appears to remove residual qSV-waves and accurately separate qP-wave data 
including the converted qS-qP waves from
the pseudo-pure-mode wavefields in this complex model.

\multiplot{4}{vp0,epsi,del,the}{width=0.3\textwidth}
{
Partial region of the 2D BP TTI model: (a) vertical qP-wave velocity, Thomsen coefficients
 (b) $\epsilon$ and (c) $\delta$, and (d) the tilt angle $\theta$. 
}

\multiplot{6}{Elasticx,Elasticz,PseudoPurePx,PseudoPurePz,PseudoPureP,PseudoPureSepP}{width=0.3\textwidth}
{
Synthesized elastic wavefields on BP 2007 TTI model using original elastic wave equation and pseudo-pure-mode 
qP-wave equation respectively: (a) x- and 
(b) z-components synthesized by original elastic wave equation; (c) x- and
 (d) z-components synthesized by pseudo-pure-mode qP-wave equation; 
 (e) pseudo-pure-mode scalar qP-wave fields; (f) separated scalar qP-wave fields.
}


%%%%%%%%%%%%%%%%%%%%%%%%%%%%%%%%%%%%%%%%%%%%%%%%%%%%%%%%%%%%%%%%%%%%%%%%%%%%%%%%%%%%%%%%%%%%%%%%%%%%%
\subsubsection{1.4 Homogeneous 3D ORT model}

\inputdir{ort3dhomo}

Figure 9 shows an example of simulating propagation of separated qP-wave fields in a 3D homogeneous
 vertical ORT model, in which $v_{p0}=3000 m/s$,
$v_{s0}=1500 m/s$, $\delta_{1}=-0.1$, $\delta_{2}=-0.0422$, $\delta_{3}=0.125$, $\epsilon_{1}=0.2$,
$\epsilon_{2}=0.067$, $\gamma_{1}=0.1$, and $\gamma_{2}=0.047$.
The first three pictures display wavefield snapshots at 0.5s synthesized by using
 pseudo-pure-mode qP-wave equation, according to equation~\ref{eq:ort}.
As shown in Figure 9d, qP-waves again appear\old{s} to dominate the wavefields in energy when we sum the 
three wavefield components of the pseudo-pure-mode qP-wave fields.
As for TI media, we obtain completely separated qP-wave fields from the
 pseudo-pure-mode wavefields once the correction of projection deviation is finished (Figure 9e).
By the way, in all above examples, we find that the filtering to remove qSV-waves does not
require the numerical dispersion of the qS-waves to be limited. So there is no additional requirement
of the grid size for qS-wave propagation. The effects of grid dispersion for the separation of low velocity
qS-waves will be further investigated in the second article of this series.
\multiplot{5}{PseudoPurePx,PseudoPurePy,PseudoPurePz,PseudoPureP,PseudoPureSepP}{width=0.3\textwidth}
{
Synthesized wavefield snapshots in a 3D homogeneous vertical ORT medium: (a) x-, (b) y- and (c) z-component
 of the pseudo-pure-mode qP-wave fields, (d) pseudo-pure-mode scalar qP-wave fields, (e) separated scalar qP-wave fields.
}


%%%%%%%%%%%%%%%%%%%%%%%%%%%%%%%%%%%%%%%%%%%%%%%%%%%%%%%%%%%%%%%%%%%%%%%%%%%%%%%%%%%%%%%%%%%%%%%%%%%%%
\subsection{2. Reverse-time migration of Hess VTI model}
\inputdir{hessvti}

Our final example shows application of the pseudo-pure-mode qP-wave equation (i.e., equation~\ref{eq:pseudoVTIxy} in its 2D form) 
 to RTM of conventional seismic data representing mainly qP-wave energy using the synthetic data of
 SEG/Hess VTI model (Figure 10).
In the original data set, there is no vertical velocity model for qSV-wave, namely $v_{s0}$.
 For simplicity, we first get this parameter by setting $\frac{v_{s0}}{v_{p0}}=0.5$ anywhere.
Figures 11a and 11b display the two components of the synthesized pseudo-pure-mode qP-wave fields,
 in which the source is located at the center of the windowed region of the original models. 
 We observe that the summed wavefields (i.e., pseudo-pure-mode scalar qP-wave fields) contain quite weak
 residual qSV-wave energy (Figure 11c).
For seismic imaging of qP-wave data, we try the finite nonzero $v_{s0}$ scheme \cite[]{fletcher:2009}
 to suppress qSV-wave artifacts and enhance computation stability.
Thanks to superposition of multi-shot migrated data, we obtain a good RTM result (Figure 12) 
using the common-shot gathers provided at http://software.seg.org, although spatial filtering
has not been used to remove the residual qSV-wave energy. This example shows that the proposed pseudo-pure-mode
qP-wave equation could be directly used for reverse-time migration of conventional single-component seismic data.

\multiplot{3}{hessvp0,hessepsilon,hessdelta}{width=0.3\textwidth}
{
Part of SEG/Hess VTI model with parameters of (a) vertical qP-wave velocity, Thomsen coefficients 
(b) $\epsilon$ and (c) $\delta$.
}

\multiplot{3}{PseudoPurePx,PseudoPurePz,PseudoPureP}
{width=0.3\textwidth}
{
Synthesized wavefields using the pseudo-pure-mode qP-wave equation in SEG/Hess VTI model:
The three snapshots are synthesized by fixing the ratio of $v_{s0}$ to $v_{p0}$ as 0.5.  
The pseudo-pure-mode qP-wave fields (c) are the sum of the (a) x- and (b) z-components 
of the pseudo-pure-mode wavefields.
}

\plot{hessrtm}{width=0.75\textwidth}
{
RTM of Hess VTI model using the pseudo-pure-mode qP-wave equation with nonzero finite $v_{s0}$.
}
